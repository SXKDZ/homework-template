\documentclass{hw}

\newcolumntype{L}{D{.}{.}{2,2}}

\course
{LaTeX Practice}
{Fall 2019}
{Local University}

\assignment
{Assignment 1}
{Get Ready for LaTeX}

\student
{Student Name}
{2019000000}
{Student Affliation}
{name.student@mail.com}

\begin{document}
	
\newproblemset{problem}{Problem}{Problems}
\newproblemset{computerexercise}{Computer Exercise}{Computer Exercises}

\maketitle

\makeproblem

\begin{problem*}[Description Goes Here]
Consider the following decision rule for a two-category one-dimensional problem: 
Decide \(\omega_1\) if \(x > \theta\); otherwise decide \(\omega_2\).

(a) Show the probability of error for this rule is given by
\[
P(\text{error}) = P\left(\omega_{1}\right) \int_{-\infty}^{\theta} p\left(x | \omega_{1}\right) \mathrm{d}x+P\left(\omega_{2}\right) \int_{\theta}^{\infty} p\left(x | \omega_{2}\right) \mathrm{d}x.
\]

(b) By differentiating, show that a necessary condition to minimize \(P(\text{error})\) is that satisfy
\[
p(\theta | \omega_1) P(\omega_1) = p(\theta | \omega_2) P(\omega_2).
\]

(c) Does this equation define \(\theta\) uniquely?

(d) Give an example where a value of \(\theta\) satisfying the equation actually maximizes the probability of error.
\end{problem*}

\begin{answer}
答案写在此处。
\end{answer}

\begin{problem}[标题]
在一个10类的模式识别问题中,有3类单独满足多类情况1,其余的类别满足多类情况2。问该模式识别问题所需判别函数的最少数目是多少?
\end{problem}

\begin{answer}
答案写在此处。
\end{answer}

\makecomputerexercise

此处可以插入一些说明。

Several of the computer exercises will rely on the following data.

\begin{computerexercise}
Illustrate the fact that the average of a large number of independent random variables will approximate a Gaussian by the following:

(a) Write a program to generate n random integers from a uniform distribution \(U(x_l, x_u)\).

(b) Now write a routine to choose \(x_l\) and \(x_u\) randomly, in the range \(-100 \leq x_l < x_u \leq 100\), and \(n\) (the number of samples) randomly in the range \(0 < n \leq 1000\).

(c) Generate and plot a histogram of the accumulation of \(10^4\) points sampled as just described.

(d) Calculate the mean and standard deviation of your histogram, and plot it.

(e) Repeat the above for \(10^5\) and for \(10^6\). Discuss your results.
\end{computerexercise}

\begin{answer}
答案写在此处。
\end{answer}

\end{document}
