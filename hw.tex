\documentclass{hw}

\course
{LaTeX Practice}
{Fall 2019}
{Local University}

\assignment
{Assignment 1}
{Get Ready for LaTeX}

\student
{Student Name}
{2019000000}
{Student Affliation}
{name.student@mail.com}

\begin{document}
	
\newproblemset{problem}{Problem}{Problems}
\newproblemset{computerexercise}{Computer Exercise}{Computer Exercises}

\maketitle

\makeproblem

\begin{problem*}[无标号问题]
在一个10类的模式识别问题中,有3类单独满足多类情况1,其余的类别满足多类情况2。问该模式识别问题所需判别函数的最少数目是多少?
\end{problem*}

\begin{answer*}
答案写在此处。
\end{answer*}

\begin{problem}

Consider the following decision rule for a two-category one-dimensional problem: 
Decide \(\omega_1\) if \(x > \theta\); otherwise decide \(\omega_2\).

(a) Show the probability of error for this rule is given by
\[
P(\text{error}) = P\left(\omega_{1}\right) \int_{-\infty}^{\theta} p\left(x | \omega_{1}\right) \mathrm{d}x+P\left(\omega_{2}\right) \int_{\theta}^{\infty} p\left(x | \omega_{2}\right) \mathrm{d}x.
\]

(b) By differentiating, show that a necessary condition to minimize \(P(\text{error})\) is that satisfy
\[
p(\theta | \omega_1) P(\omega_1) = p(\theta | \omega_2) P(\omega_2).
\]

(c) Does this equation define \(\theta\) uniquely?

(d) Give an example where a value of \(\theta\) satisfying the equation actually maximizes the probability of error.

\end{problem}

\begin{answer}
考虑分类错误率的条件概率:
\begin{equation}
	P(\text{error} | x) =
	\begin{cases}
		p(\omega_1 | x), & \text{if we decide }\omega_2, \\
		p(\omega_2 | x), & \text{otherwise.}
	\end{cases}
	\label{eq:error-conditional-probability}
\end{equation}

根据式 (\ref{eq:error-conditional-probability}) 对\(x\)进行积分,可得:
\begin{align}
	P(\text{error}) & = \int_{-\infty}^\infty P(\text{error} | x) p(x) \mathrm{d} x \\
	& = P(x \leq \theta , x \text{ is } \omega_1 ) + P(x > \theta , x \text{ is } \omega_2 ) \\
	& = p(x \leq \theta | \omega_1) P(\omega_1) + p(x > \theta | \omega_2) P(\omega_2) \\
	& = P\left(\omega_{1}\right) \int_{-\infty}^{\theta} p\left(x | \omega_{1}\right) \mathrm{d} x + P\left(\omega_{2}\right) \int_{\theta}^{\infty} p\left(x | \omega_{2}\right) \mathrm{d} x. \label{eq:error-probability}
\end{align}
\end{answer}

\makecomputerexercise

Several of the computer exercises will rely on the following data.

此处还可以插入一些说明。

\begin{computerexercise}
Illustrate the fact that the average of a large number of independent random variables will approximate a Gaussian by the following:

(a) Write a program to generate n random integers from a uniform distribution \(U(x_l, x_u)\).

(b) Now write a routine to choose \(x_l\) and \(x_u\) randomly, in the range \(-100 \leq x_l < x_u \leq 100\), and \(n\) (the number of samples) randomly in the range \(0 < n \leq 1000\).

(c) Generate and plot a histogram of the accumulation of \(10^4\) points sampled as just described.

(d) Calculate the mean and standard deviation of your histogram, and plot it.

(e) Repeat the above for \(10^5\) and for \(10^6\). Discuss your results.
\end{computerexercise}

\begin{answer}
答案写在此处,如代码~\ref{lst:python}~所示。

\begin{lstlisting}[language=python,caption=代码测试,label=lst:python]
print('Hello, world!')
\end{lstlisting}

为了得到\(p(x | \omega) \sim \mathcal{N}(0, 1)\):
\begin{equation}
	p(x | \omega_i) = \frac{1}{\sqrt{2 \pi} \sigma}\exp{\left[- \frac{1}{2} \left(\frac{x - \mu}{\sigma}\right)^2\right]}
	\label{eq:Gaussian-distribution}
\end{equation}

对式 (\ref{eq:Gaussian-distribution}),我们可以进行一些计算。

\end{answer}

\end{document}
